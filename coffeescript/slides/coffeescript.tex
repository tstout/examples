%
% - Presentation Plan -
% * History
% * Maybe - mention several of the JS warts CS works around
% * Basic Operators
% * Namespaces?
% * Object literal sugar
% * Destructuring
% * String interp.
% * Classes            - Might show this live with js to coffee
% * Fat arrow
% * Method default args
% * Testing options - Show github mocha + chai?
% * Show merchinator code using JS libs
% * Show debugging merchinator code in chrome
% * Pick existing JS code and show CS version (Dashboards repo is
%   probably a good source)
%
% Compile with 'pdflatex -shell-escape coffeescript.tex'
%
\documentclass{beamer}

\usetheme{Warsaw}    

\usepackage{listings}                                   
\usepackage{hyperref}
\usepackage{graphicx}                                 
\usepackage{tabularx}
\usepackage{microtype}
\usepackage[T1]{fontenc}
\usepackage[scaled]{beramono}
\usepackage{minted}
\usepackage{multicol}

\usepackage{xcolor}

\newcommand\Small{\fontsize{5}{5.2}\selectfont}
\newcommand*\LSTfont{\Small\ttfamily\SetTracking{encoding=*}{-60}\lsstyle}

\hypersetup{colorlinks=color, linkcolor=black}
\definecolor{OliveGreen}{rgb}{0,0.6,0}
\graphicspath{{./images/}}
%
% Turn off beamer nav stuff...
%
\setbeamertemplate{navigation symbols}{}

%\input{lst-config/clojure-config}
\begin{document}

\begin{frame}
  \frametitle{Intro to}
  \center{
    \includegraphics[scale=.60]{cs-logo}
  }

  April 2014

\end{frame}

\begin{frame}
  \frametitle{History}
  \includegraphics[scale=.60]{jeremy}  
  \begin{itemize}
    \item December 2009 - Jeremy Ashkenas first git commit
      \pause
    \item First version written in Ruby
      \pause
    \item February 2010 - Ruby version replaced by self-hosting
      implementation
      \pause
    \item Jeremy also created Backbone and Underscore
  \end{itemize}  
\end{frame}



\begin{frame}
  \frametitle{Why CoffeeScript?}
  \begin{itemize}
    \pause
    \item A language that takes out the frustrating and overly verbose bits of JS, and provides a safer, briefer way to stick to the good parts.
      \pause
    \item Syntax
    \pause
    \item ES6 Now!
  \end{itemize}
\end{frame}

\begin{frame}
  \frametitle{Why Not?}
  \begin{itemize}
    \pause
    \item Syntax
    \pause
    \item Abstraction (Not another GWT!)
    \pause
    \item Compilation
    \pause
    \item Debugging
      \pause
  \end{itemize}
\end{frame}


% \inputminted{csharp}{hello.cs}

\begin{frame}
  \frametitle{Plenty of Sugar}
  \includegraphics[scale=.40]{sugar}
%
% TODO - consider using columns for better appearance
%
  \begin{itemize}
    \item Multi-line strings (heredoc)
    \item String interpolation
    \item Default arguments
    \item Existential operator
    \item Splats
    \item Object literals
    \item Fat arrow
    \item Prototype and this alias
    \item Classes
  \end{itemize}
\end{frame}

\begin{frame}[fragile]
  \frametitle{Function Definitions}

  \begin{minted}[lineos=false,fontsize=\normalsize]{coffeescript}
    inc = (a) ->
      a + 1
  \end{minted}

  \pause
  \vspace{.5cm}
  Compiles to:
  \vspace{.5cm}

  \begin{minted}[lineos=false,fontsize=\normalsize]{javascript}
    var inc;

    inc = function(a) {
      return a + 1;
    };    
  \end{minted}  
\end{frame}

\begin{frame}[fragile]
  \frametitle{String Interpolation}
 
  \begin{minted}[linenos=false,fontsize=\normalsize]{coffeescript}
    "#{name} is #{age} years old."
  \end{minted}

  \pause
  \vspace{.5cm}
  Compiles to:
  \vspace{.5cm}

  \begin{minted}[linenos=false,fontsize=\normalsize]{javascript}
    "" + name + " is " + age + " years old."
  \end{minted}  
\end{frame}

\begin{frame}
  \frametitle{Default Arguments}
  
\end{frame}

\begin{frame}
\frametitle{Variable Arguments}

\end{frame}

%
% ---- Testing for Existence ----
%
 \begin{frame}[fragile]
  \frametitle{Existential Operator}


  Testing for existence:

  \begin{minted}[lineos=false,fontsize=\normalsize]{coffeescript}
    if variable?
      console.log('variable is declared and not null')
  \end{minted}

  \pause
  \vspace{.5cm}
  Compiles to:
  \vspace{.5cm}


  \begin{minted}[lineos=false,fontsize=\normalsize]{javascript}
    if (typeof variable !== "undefined" && 
        variable !== null) {
      console.log('variable is declared and not null');
    }
  \end{minted}  
\end{frame}

%
% ----- Conditional Assigment -------
%
 \begin{frame}[fragile]
  \frametitle{Existential Operator}


  Conditional Assignment:

  \begin{minted}[lineos=false,fontsize=\normalsize]{coffeescript}
    getUserProfile = ->
      @profile ?= DB.getProfile(User.current)
  \end{minted}

  \pause
  \vspace{.5cm}
  Compiles to:
  \vspace{.5cm}


  \begin{minted}[lineos=false,fontsize=\normalsize]{javascript}
    getUserProfile = function() {
      return this.profile != null ? this.profile : 
        this.profile = DB.getProfile(User.current);
    };
  \end{minted}  
\end{frame}

 \begin{frame}[fragile]
  \frametitle{Existential Operator}


  Field Chaining:

  \begin{minted}[lineos=false,fontsize=\normalsize]{coffeescript}
    zip = User.current?.address?.zip
  \end{minted}

  \pause
  \vspace{.5cm}
  Compiles to:
  \vspace{.5cm}

  \begin{minted}[lineos=false,fontsize=\normalsize]{javascript}
    var zip, _ref, _ref1;

    zip = (_ref = User.current) != null ? 
      (_ref1 = _ref.address) != null ? _ref1.zip : 
      void 0 : void 0;
  \end{minted}  
\end{frame}

 \begin{frame}[fragile]
  \frametitle{Existential Operator}

  Conditional function invocation:

  \begin{minted}[lineos=false,fontsize=\normalsize]{coffeescript}
    someFunction?()
  \end{minted}

  \pause
  \vspace{.5cm}
  Compiles to:
  \vspace{.5cm}

  \begin{minted}[lineos=false,fontsize=\normalsize]{javascript}
    if (typeof someFunction === "function") {
      someFunction();
    }
  \end{minted}  
\end{frame}

\begin{frame}
  \frametitle{Heredoc}
  \inputminted{coffeescript}{src/heredoc.coffee}
\end{frame}

\begin{frame}[fragile]
  \frametitle{Embedding Raw JavaScirpt}
  \begin{minted}[lineos=false,fontsize=\Large]{coffeescript}
    rawJS = `function() {
      return someComplexThing;
    }`
  \end{minted}
\end{frame}


\begin{frame}
  \frametitle{Tooling}
    \includegraphics[scale=.40]{tooling}
  \begin{multicols}{2}
  \begin{itemize}
    \item Intellij/Rubymine
    \item Emacs
    \item Vim
    \item Sublime
    \item node
    \item \href{http://www.coffeelint.org/}{CoffeLint}
    \item Rails
    \item \href{http://www.playframework.com/documentation/2.0/AssetsCoffeeScript}{Play}
    \item Eclipse
    \item \href{https://jawr.java.net/}{Jawr}
    \item
      \href{http://www.html5rocks.com/en/tutorials/developertools/sourcemaps/}{Source
      Maps}

  \end{itemize}
\end{multicols}
\end{frame}

\end{document}
