\documentclass{beamer}

\usetheme{Warsaw}    

\usepackage{listings}                                   
\usepackage{hyperref}
\usepackage{graphicx}                                 
\usepackage{tabularx}
\usepackage{microtype}
\usepackage[T1]{fontenc}
\usepackage[scaled]{beramono}

\usepackage{xcolor}

\newcommand\Small{\fontsize{5}{5.2}\selectfont}
\newcommand*\LSTfont{\Small\ttfamily\SetTracking{encoding=*}{-60}\lsstyle}

\hypersetup{colorlinks=color, linkcolor=black}
\definecolor{OliveGreen}{rgb}{0,0.6,0}
\graphicspath{{./images/}}
%
% Turn off beamer nav stuff...
%
\setbeamertemplate{navigation symbols}{}

%\input{lst-config/clojure-config}
\begin{document}

\begin{frame}
  \frametitle{CoffeeScript Intro}
  \center{
    \includegraphics[scale=.60]{cs-logo}
  }

  April 2014

\end{frame}

\begin{frame}
  \frametitle{Why CoffeeScript?}
  \begin{itemize}
    \pause
    \item CoffeeScript is an attempt to expose the good parts of
      JavaScript in a simple way.
      \pause
    \item Concise
      \pause
    \item Remove typical Javascript boilerplate (semicolons, braces,
      function keyword)
      \pause
    \item  typically 1/3 fewer lines, with no effect on runtime performance
  \end{itemize}
\end{frame}

\begin{frame}
  \frametitle{History}
  \begin{itemize}
    \item December 2009 - Jeremy Ashkenas made first Git commit
    \pause
    \item Inspired by many languages including Ruby, Python, Perl, and Haskell
      \pause
    \item First version written in Ruby
      \pause
    \item February 2010 - Ruby version replaced by self-hosting implementationx
  \end{itemize}  
\end{frame}


\end{document}
