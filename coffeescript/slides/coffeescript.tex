%
% - Presentation Plan -
% * History
% * Maybe - mention several of the JS warts CS works around
% * Basic Operators
% * Namespaces?
% * Object literal sugar
% * Destructuring
% * String interp.
% * Classes            - Might show this live with js to coffee
% * Fat arrow
% * Method default args
% * Testing options - Show github mocha + chai?
% * Show merchinator code using JS libs
% * Show debugging merchinator code in chrome
% * Pick existing JS code and show CS version (Dashboards repo is
%   probably a good source)
%
% Compile with 'pdflatex -shell-escape coffeescript.tex'
%
\documentclass{beamer}

\usetheme{Warsaw}    

\usepackage{listings}                                   
\usepackage{hyperref}
\usepackage{graphicx}                                 
\usepackage{tabularx}
\usepackage{microtype}
\usepackage[T1]{fontenc}
\usepackage[scaled]{beramono}
\usepackage{minted}

\usepackage{xcolor}

\newcommand\Small{\fontsize{5}{5.2}\selectfont}
\newcommand*\LSTfont{\Small\ttfamily\SetTracking{encoding=*}{-60}\lsstyle}

\hypersetup{colorlinks=color, linkcolor=black}
\definecolor{OliveGreen}{rgb}{0,0.6,0}
\graphicspath{{./images/}}
%
% Turn off beamer nav stuff...
%
\setbeamertemplate{navigation symbols}{}

%\input{lst-config/clojure-config}
\begin{document}

\begin{frame}
  \frametitle{Intro to}
  \center{
    \includegraphics[scale=.60]{cs-logo}
  }

  April 2014

\end{frame}

\begin{frame}
  \frametitle{Why CoffeeScript?}
  \begin{itemize}
    \pause
    \item A language that takes out the frustrating and overly verbose bits of JS, and provides a safer, briefer way to stick to the good parts.
      \pause
    \item  Typically 1/3 fewer lines, with no effect on runtime performance
  \end{itemize}
\end{frame}

\begin{frame}
  \frametitle{History}
  \includegraphics[scale=.60]{jeremy}  
  \begin{itemize}
    \item December 2009 - Jeremy Ashkenas first git commit
      \pause
    \item First version written in Ruby
      \pause
    \item February 2010 - Ruby version replaced by self-hosting
      implementation
      \pause
    \item Jeremy also created Backbone and Underscore
  \end{itemize}  
\end{frame}

% \inputminted{csharp}{hello.cs}

\begin{frame}
  \frametitle{Plenty of Sugar}
  \includegraphics[scale=.40]{sugar}
%
% TODO - consider using columns for better appearance
%
  \begin{itemize}
    \item Multi-line strings (heredoc)
    \item String interpolation
    \item Default arguments
    \item Existential operator
    \item Splats
    \item Object literals
    \item Fat arrow
    \item prototype and this alias
  \end{itemize}
\end{frame}

\begin{frame}
  \frametitle{Heredoc}
  \inputminted{coffeescript}{src/heredoc.coffee}
\end{frame}

\begin{frame}[fragile]
  \frametitle{String Interpolation}
  CS Version
  \begin{minted}[linenos=false,fontsize=\small]{coffeescript}
    "[id^=edit_#{subject}_],[id^=new_#{subject}]"
  \end{minted}
  JS Version
  \begin{minted}[linenos=false,fontsize=\small]{javascript}
    "[id^=edit_" + subject + "_],[id^=new_" + subject + "]";
  \end{minted}  
\end{frame}

\begin{frame}
  \frametitle{Default Arguments}
  
\end{frame}

\begin{frame}
\frametitle{Variable Arguments}

\end{frame}

\begin{frame}
  \frametitle{Useful Operators}
  \begin{itemize}
    \item ?= is the existential operator. It checks against null and undefined
    \item ||= checks the variable's value to be "not false"
   \end{itemize}
\end{frame}


\begin{frame}[fragile]
\begin{minted}[linenos=false,
  fontsize=\tiny]{coffeescript}
#
# Don't allow duplicates between primary and recommended skus...
#
skuFilter = (parsedResponse) ->
  _.intersection(
    duplicateSkuFilter(parsedResponse, '.sku-list-cell'),
    duplicateSkuFilter(parsedResponse, '.recommended-sku-list-cell'))
\end{minted}
\end{frame}
\end{document}
