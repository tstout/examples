\documentclass{beamer}

\usetheme{Warsaw}    
%\usetheme{Dresden}                  % looks pretty good
%\usetheme{Frankfurt}                % +1
%\usetheme{Goettingen}             % +1 simple and clean
%\usetheme{Ilmenau}                    % +1 bright
%\usetheme{Luebeck}                 % +1 can't tell much diff from warsaw
%\usetheme{Madrid}                   % +1 like warsaw, but date footer
%\usetheme{Malmoe}                  % Like warsaw, but not header colors
%\usetheme{Singapore}               % +1 minimal and clean
                        
\usepackage{tikz}                                         % SM Diagramming stuff
\usepackage{listings}                                   % Source code formatting
\usepackage{hyperref}
\usepackage{graphicx}                                 % Table resizing
\usepackage{tabularx}
\usepackage{microtype}
\usepackage[T1]{fontenc}
\usepackage[scaled]{beramono}


%\newfontfamily\listingsfont[Scale=0.7]{Courier} 
%\newfontfamily\listingsfontinline[Scale=0.8]{Courier New} 

\newcommand\Small{\fontsize{5}{5.2}\selectfont}
\newcommand*\LSTfont{\Small\ttfamily\SetTracking{encoding=*}{-60}\lsstyle}

\usetikzlibrary{automata,positioning}
\hypersetup{colorlinks=true,linkcolor=blue}

%\newfontfamily\listingsfont[Scale=0.7]{Courier} 
%\newfontfamily\listingsfontinline[Scale=0.8]{Courier New}

%  basicstyle={\ttfamily\tiny},

%
% Research Notes:
%
% Here's some intro to SM from stateworks
% http://www.stateworks.com/active/download/wagf04-2-state-machine-misunderstandings.pdf
%
% Here's the book from stateworks:
% http://is.ifmo.ru/download/modelingsoftwarewithfinitestatemachinesapracticalapproach.pdf

% Here's an example of using metapost uml:
% http://rxwen.blogspot.com/2010/11/draw-uml-with-latex.html
%

\begin{document}
%
% Define varibles here for simplifying the document (in particular tables)
%
\newcommand{\smcUrl} {
  \href{http://smc.sourceforge.net/}{smc}
}

\newcommand{\ragelUrl} {
  \href{http://www.complang.org/ragel/}{ragel}
} 

\newcommand{\machinaUrl} {
  \href{https://github.com/ifandelse/machina.js}{machina}
}

\newcommand{\jssmUrl} {\href{https://github.com/jakesgordon/javascript-state-machine}{js-sm} 
}

\newcommand{\squirrelUrl} {
  \href{https://github.com/hekailiang/squirrel}{squirrel}
}

\newcommand{\easyFlowUrl} {
  \href{https://github.com/Beh01der/EasyFlow/}{easyflow}
}

\newcommand{\statelessUrl} {
  \href{https://code.google.com/p/stateless4j/}{stateless4j}
}

%
% Title Page
%
  \begin{frame}
    \frametitle{State Machines}

    \begin{center}
      \begin{tikzpicture}[shorten >=1pt,node distance=2.75cm,on grid,auto] 
        \node[state,initial] (s_0)   {$s_0$}; 
        \node[state] (s_1) [above right=of s_0] {$s_1$}; 
        \node[state] (s_2) [below right=of s_0] {$s_2$}; 
        \node[state](s_3) [below right=of s_1] {$s_3$};
        \path[->] 
        (s_0) edge  node {0} (s_1)
        edge  node [swap] {1} (s_2)
        (s_1) edge  node  {1} (s_3)
        edge [loop above] node {0} ()
        (s_2) edge  node [swap] {0} (s_3) 
        edge [loop below] node {1} ();
      \end{tikzpicture}
      
    \end{center}

November 2013

\end{frame}

%
% Introduction
%
\frame{
  \frametitle{Applications}
  \itemize{
%
% TODO - need more detailed examples...
%
  \item Simple parsers and lexers\pause
  \item Communication Protocols \pause
  \item Complex User Inerfaces \pause
  \item Validating User Input \pause
  \item Digital Circuits
   }
}
  
%
% Why/When/How
%
\frame{
    \frametitle{Why?}
\begin{itemize}
  \item Manage and simplify complexity
    \pause
   \item  Specify intent and implementation separately
\end{itemize}
}

\begin{frame}
  \frametitle{When?}
  A state machine might be appropriate if
  \begin{itemize}
    \pause
    \item many flag variables
     \pause
    \item deeply nested if statements
  \end{itemize} 
\end{frame}

%http://media.pragprxog.com/articles/nov_02_state.pdf

\begin{frame}[fragile]
  \frametitle{How?}
\begin{lstlisting}
for (int i = 0; i < 20; i++) {
  // do something
}
\end{lstlisting}

\end{frame}

\begin{frame}
  \frametitle{SMC Distinguishing Features}
  
\end{frame}

%\lstset{ columns=fullflexible with basicstyle=\tiny\sffamily, language=ruby}
\lstset{frame=tb,
  language=Ruby,
  aboveskip=1mm,
  belowskip=1mm,
  showstringspaces=false,
  keepspaces=true
  columns=fullflexible,
  basicstyle={\LSTfont},
  numbers=none,
  numberstyle=\tiny\color{gray},
  keywordstyle=\color{blue},
  commentstyle=\color{orange},
  stringstyle=\color{mauve},
  breaklines=false,
  breakatwhitespace=false
  tabsize=2
}
\begin{frame}[fragile]
  \frametitle{How?}
Ruby Matrix Example
%\begin{lstlisting}
\lstinputlisting{ruby_state_matrix.rb} 
%\end{lstlisting}

\end{frame}



\begin{frame}
  \frametitle{SMC Distinguishing Features}
  
\end{frame}





%
% As a side note, look at the complexity of this:
% http://www.w3.org/TR/scxml/
% Wow, simply wow. 
%

\begin{frame}
  \frametitle{State Machine Frameworks (small sampling)}
  \begin{tabularx}{\textwidth}{ |X|X|X|X|X| }
    \hline
    & Language Support & External DSL & Internal DSL & Open Source \\
    \hline
    \statelessUrl & Java & & X & \\ 
    \hline
    \smcUrl  & Numerous & X & & \\ 
    \hline
     \ragelUrl & Several & X & & \\ 
    \hline
     \easyFlowUrl & Several & X & & \\ 
    \hline
   \squirrelUrl & Several & X & & \\ 
    \hline
    \machinaUrl &  javascript & & X & \\
    \hline
    \jssmUrl & javascript & & X & \\
    \hline
    \end{tabularx}
\end{frame}


\end{document}
