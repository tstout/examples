% 
% Compile with 'pdflatex -shell-escape clojure-basics.tex'
% I use the mactex distro of Tex. https://tug.org/mactex/
% 

\documentclass{beamer}

\usetheme{Madrid}    

\usepackage{listings}                                   
\usepackage{hyperref}
\usepackage{graphicx}                                 
\usepackage{tabularx}
\usepackage{microtype}
\usepackage[T1]{fontenc}
\usepackage[scaled]{beramono}
\usepackage{minted}
\usepackage{xcolor}
\usepackage{pgfplots}
\usepackage{dirtytalk}
\usepackage{tikz}
\usetikzlibrary{tikzmark,fit}

%\usepackage{enumitem}
\pgfplotsset{compat=1.6} 

\newcommand\Small{\fontsize{5}{5.2}\selectfont}
\newcommand*\LSTfont{\Small\ttfamily\SetTracking{encoding=*}{-60}\lsstyle}
\renewcommand{\footnotesize}{\tiny}

\hypersetup{colorlinks=color, linkcolor=black}
\definecolor{OliveGreen}{rgb}{0,0.6,0}
\graphicspath{{./images/}}
% 
% Turn off beamer nav stuff...
% 
\setbeamertemplate{navigation symbols}{}


%\input{lst-config/clojure-config}
\begin{document}

\begin{frame}
  \frametitle{<Your Title Here>}
  \center{
    %
    % Graphic for Title Page
    %\includegraphics[scale=.40]{Clojure-Logo}
    %
  }
\end{frame}

\frame{
  \frametitle{Your Title Page 1}
  \say{Optional Catchy quote\ldots}

  \rightline{{\rm  Author of Quote}}
}

\frame{
  \frametitle{Title Page 2}
  % \begin{columns}
  %   \begin{column}{.49\textwidth}
  %     \includegraphics[scale=.50]{church}
  %   \end{column}
  %   \begin{column}{.49\textwidth}
  %     \itemize{
  %     \item Alonzo Church - 1936
  %     \item $\lambda$ Calculus
  %      \footnote{\href{https://www.ics.uci.edu/~lopes/teaching/inf212W12/readings/church.pdf}
  %        {An Unsolvable Problem of Elementary Number Theory}}
  %     }

  %   \end{column}
  %\end{columns}
}

%\resetcounter[footnote]
%\setcounter{footnote}{0} 

\frame{
  \frametitle{Title Page 3}
  % \begin{columns}
  %   \begin{column}{.49\textwidth}
  %     \includegraphics[width=\textwidth,height=\textheight,keepaspectratio=true]{john-mccarthy}
  %   \end{column}
  %   \begin{column}{.49\textwidth}
  %     % \vspace {4 cm}
  %     \itemize{
  %     \item John McCarthy - 1958
  %     \item
  %       OriginalPaper - 1960
  %       \footnote{\href{http://www-formal.stanford.edu/jmc/recursive.html}{Original Paper}}
  %     \item 
  %       Influenced by Church \& IPL 
  %       \footnote{\href{https://en.wikipedia.org/wiki/Information_Processing_Language}{
  %           IPL Overview}}
  %       \vspace {4 cm}
  %     }
  %   \end{column}
  % \end{columns}
}

% \frame{
%   \frametitle{Title - Page of Items}
%   \begin{itemize}
%   \item Item 1
%   % To not show following item until keypress
%   % \pause
%   \item Item 2
%   \item Item 3
%   \end{itemize}
% }

% Example of inlining code with minted
% \begin{frame}[fragile]
%   \frametitle{Functions}
%   Typical function definition:
%   \begin{minted}[fontsize=\normalsize,escapeinside=||]{clojure}
%     (defn foo 
%       "optional multiline doc comment"
%       {:optional :metadata-map} 
%       [arg1 arg2]
%       {:optional :pre/post-conditions-map}
%       (...implementation...))
%   \end{minted}

%   \vspace{1 cm}

% %% [autogobble,fontfamily=myfont,escapeinside=||}{c}

%   This is an anonymous function (sum of its args):
%   \begin{minted}[fontsize=\normalsize]{clojure}
%     (fn [a b] (+ a b))   
%   \end{minted}
%   \vspace{1 cm}

%   An abbreviated equivalent:
%   \begin{minted}[fontsize=\normalsize]{clojure}
%     #(+ %1 %2)
%   \end{minted}
% \end{frame}


%https://clojure.org/api/cheatsheet
\frame{
  \frametitle{Resources}
  \begin{itemize}

  % \item \href {https://www.example.com/}
  %   {\color {blue}{Link to useful info}}

    %% put back ref to source code-[]
  \end{itemize}
}

\end{document}
